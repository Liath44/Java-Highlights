\documentclass[]{article}
\usepackage{lmodern}
\usepackage{blindtext}
\usepackage{amssymb,amsmath}
\usepackage{ifxetex,ifluatex}
\usepackage[document]{ragged2e}
\usepackage{fixltx2e} % provides \textsubscript
\ifnum 0\ifxetex 1\fi\ifluatex 1\fi=0 % if pdftex
\usepackage[T1]{fontenc}
\usepackage[utf8]{inputenc}
\else % if luatex or xelatex
\ifxetex
\usepackage{mathspec}
\else
\usepackage{fontspec}
\fi
\defaultfontfeatures{Ligatures=TeX,Scale=MatchLowercase}
\fi
% use upquote if available, for straight quotes in verbatim environments
\IfFileExists{upquote.sty}{\usepackage{upquote}}{}
% use microtype if available
\IfFileExists{microtype.sty}{%
\usepackage{microtype}
\UseMicrotypeSet[protrusion]{basicmath} % disable protrusion for tt fonts
}{}
\usepackage[unicode=true]{hyperref}
\hypersetup{
pdfborder={0 0 0},
breaklinks=true}
\urlstyle{same}  % don't use monospace font for urls
\IfFileExists{parskip.sty}{%
\usepackage{parskip}
}{% else
\setlength{\parindent}{0pt}
\setlength{\parskip}{6pt plus 2pt minus 1pt}
}
\setlength{\emergencystretch}{3em}  % prevent overfull lines
\providecommand{\tightlist}{%
\setlength{\itemsep}{0pt}\setlength{\parskip}{0pt}}
\setcounter{secnumdepth}{0}
% Redefines (sub)paragraphs to behave more like sections
\ifx\paragraph\undefined\else
\let\oldparagraph\paragraph
\renewcommand{\paragraph}[1]{\oldparagraph{#1}\mbox{}}
\fi
\ifx\subparagraph\undefined\else
\let\oldsubparagraph\subparagraph
\renewcommand{\subparagraph}[1]{\oldsubparagraph{#1}\mbox{}}
\fi
\title{Specyfikacja funkcjonalna programu JVac}
\date{}

\begin{document}

    \maketitle


    \section{1 ) Cel}

    Celem projektu jest wyznaczenie dziennego optymalnego planu zakupu
    szczepionek dla zadanej grupy aptek od ich producentów. Producenci
    szczepionek są w stanie dziennie wyprodukować wyłącznie zadaną ich
    ilość. Dodatkowo każda apteka zamawiać może szczepionki jedynie po cenie
    ustalonej indywidualnie z każdym z producentów i w zadanej maksymalnej
    ich ilości.


    \section{2 ) Szczegóły techniczne i wymagania}

    Program napisany został w Javie i wymaga maszyny wirtualnej tegoż języka
    (zalecana wersja: 11)


    \section{3 ) Kompilacja i uruchomienie}

    Przed pierwszym użyciem program powinien zostać skompilowany. W tym celu
    użytkownik powinien przejść do katalogu \emph{src} i wpisać w wierszu poleceń/konsoli następującą
    komendę

    \begin{center}
        \emph{javac JVac.java}
    \end{center}

    Program może zostać uruchomiony następującą komendą

    \begin{center}
        \emph{java JVac ŚCIEŻKA\_DO\_PLIKU\_WEJŚCIOWEGO}
    \end{center}


    \section{4 ) Działanie programu}

    Po uruchomieniu programu dalsza ingerencja użytkownika nie jest wymagana.
    Po skończonym czasie zwracana jest wiadomość podsumowująca jego działanie (pomyślne lub nie).


    \section{5 ) Dane wejściowe}

    Program wymaga od użytkownika danych wejściowych zawartych w pliku o
    następującej postaci:\linebreak

    \# Producenci szczepionek (id \textbar{} nazwa \textbar{} dzienna
    produkcja)

    0 \textbar{} Nazwa\_producenta1 \textbar{} Dzienna\_produkcja1

    1 \textbar{} Nazwa\_producenta2 \textbar{} Dzienna\_produkcja2

    (\ldots{})

    \# Apteki (id \textbar{} nazwa \textbar{} dzienne zapotrzebowanie)

    0 \textbar{} Nazwa\_apteki1 \textbar{} Dzienne\_zapotrzebowanie1

    1 \textbar{} Nazwa\_apteki2 \textbar{} Dzienne\_zapotrzebowanie2

    (\ldots{})

    \# Połączenia producentów i aptek (id producenta \textbar{} id apteki
    \textbar{} dzienna maksymalna liczba dostarczanych szczepionek
    \textbar{} koszt szczepionki {[}zł{]} )

    Id\_producenta1 \textbar{} Id\_apteki1 \textbar{} Dzienne\_szczepionki1
    \textbar{} Cena\_sztuki1

    Id\_producenta2 \textbar{} Id\_apteki2 \textbar{} Dzienne\_szczepionki2
    \textbar{} Cena\_sztuki2

    (\ldots{})\linebreak

    Dodatkowo dane powinny spełniać następujące warunki:

    \begin{itemize}
        \item
        Nagłówki poszczególnych sekcji powinny mieć zadaną treść tj. włącznie
        z zawartością pomiędzy nawiasami
        \item
        ID aptek i producentów powinny być kolejnymi liczbami naturalnymi
        \item
        Dzienna produkcja powinna być liczbą naturalną, mogącą wynosić 0
        \item
        Dzienne zapotrzebowanie powinno być liczbą naturalną, mogącą wynosić 0
        \item
        Każdy producent powinien mieć połączenie z każdą apteką
        \item
        Dzienne szczepionki powinny być liczbą naturalną, mogącą wynosić 0
        \item
        Cena sztuki powinna być ułamkiem dziesiętnym, mogącym mieć wartość 0 i
        mającym maksymalnie 2 miejsca po przecinku
        \item
        Kolejne sekcje danych powinny znajdować się w osobnych liniach
        \item
        Poszczególne dane powinny być oddzielone znakiem `\textbar{}' --
        spacje nie są wymagane
        \item
        Białe znaki przed i po poszczególnych danych mogą być w dowolnych ilościach.
        Wiodące i zawodzące białe znaki nie są rozpatrywane przez program
        \item
        Nazwy aptek i producentów mogą zawierać spacje
        \item
        Kolejność sekcji danych powinna być zgodna z zadanym wzorem
        \item
        Nagłówek sekcji danych zawsze rozpoczyna się od pojedynczego `\#'
        \item
        Plik musi zawierać wszystkie 3 wymienione nagłówki (mogą być puste)
        \item
        Połączenia, nazwy producentów i aptek nie powinny się powtarzać
        \item
        Wartości nie powinny mieć podanych jednostek
        \item
        Ilość aptek nie powinna przekraczać 1000
        \item
        Ilość producentów nie powinna przekraczać 1000
        \item
        Dane reprezentujące liczby nie zawierają w sobie żadnych białych znaków. Dla ułamków dziesiętnych separatorem jest '.'
        \item
        Dane reprezentujące słowa powinny być podawane w Unikodzie
    \end{itemize}


    \section{6 ) Dane wyjściowe}

    Poprawne działanie programu zakończone jest odpowiednim komunikatem w
    konsoli oraz generacją pliku wyjściowego o następującym formacie:\linebreak

    Nazwa\_apteki\_X -\textgreater{} Nazwa\_producenta\_X {[}Koszt =
    Ilość\_X * Cena\_X = Wynik\_X{]}

    Nazwa\_apteki\_Y -\textgreater{} Nazwa\_producenta\_Y {[}Koszt =
    Ilość\_Y * Cena\_Y = Wynik\_Y{]}

    (\ldots{} Analogicznie dla każdej zaplanowanej transakcji \ldots{})

    Opłaty całkowite: SUMA\_OPŁAT\linebreak

    Plik wygenerowany zostanie w folderze zawierającym pliki programu


    \section{7 ) Sytuacje niepożądane}

    W przypadku błędów z kompilacją należy upewnić się, że Java 11 została
    zainstalowana na urządzeniu. W razie dalszych błędów proszę skontaktować
    się z autorem programu

    Możliwymi są również błędy w trakcie działania programu. Użytkownik jest
    o takich sytuacjach informowany odpowiednim komunikatem w konsoli.
    Poniżej lista potencjalnych powodów ku takiemu zdarzeniu:

    \begin{itemize}
        \item
        Podany plik wejściowy nie istnieje, nie może zostać otworzony, jest
        uszkodzony lub ma błędny format
        \item
        Dostarczone programowi dane są zbyt duże lub nie pozwalają na
        stworzenie jakiegokolwiek planu
        \item
        Plik wyjściowy nie mógł zostać utworzony
    \end{itemize}

    W razie wątpliwości co do przyczyny przerwania pracy programu zachęcam
    do kontaktu z autorem

    \section{8 ) Skrótowa informacja o wykorzystywanych algorytmach i
    strukturach danych}

    Program implementuje algorytm sympleksowy z cięciem Gomory'ego --
    implikuje to również obecność macierzowej struktury danych zawierającej
    metody przydatne dla tego algorytmu, algorytmu eliminacji Gaussa-Jordana, klasy
    reprezentującej liczby wymierne wraz z algorytmem Euklidesa do
    znajdowania wspólnych mianowników

    \begin{flushright}
        \emph{Maciej Dragun\\*}
        \emph{298748 WE PW\\*}
        \emph{01142044@pw.edu.pl\\*}
    \end{flushright}

\end{document}

